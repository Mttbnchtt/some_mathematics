%%%%%%%%%%%%%%%%%%%%%%%%%%%%%%%%%%%%%%%%%%%%%%%%%%%%%%%%%%%%%%%%%%%%%
% Math Book Template in LaTeX
%%%%%%%%%%%%%%%%%%%%%%%%%%%%%%%%%%%%%%%%%%%%%%%%%%%%%%%%%%%%%%%%%%%%%
\documentclass[11pt,oneside]{book}  
% The 'book' document class is ideal for longer works with chapters.
% The options '11pt' sets the font size, and 'oneside' is for printing on one side.

%%%%%%%%%%%%%%%%%%%%%%%%%%%%%%%%%%%%%%%%%%%%%%%%%%%%%%%%%%%%%%%%%%%%%
% Preamble: Packages and Customizations
%%%%%%%%%%%%%%%%%%%%%%%%%%%%%%%%%%%%%%%%%%%%%%%%%%%%%%%%%%%%%%%%%%%%%
\usepackage[utf8]{inputenc}    % Allows UTF-8 encoded characters.
\usepackage[T1]{fontenc}       % Use modern font encoding.
\usepackage{lmodern}           % Use Latin Modern fonts.

% AMS packages for enhanced math typesetting:
\usepackage{amsmath, amssymb, amsthm}  

% For including graphics:
\usepackage{graphicx}

% For hyperlinks in the document:
\usepackage{hyperref}
\hypersetup{
    colorlinks=true,
    linkcolor=blue,     % Color for internal links
    citecolor=blue,     % Color for bibliographic citations
    urlcolor=blue       % Color for URLs
}

% Cleveref provides intelligent cross-referencing:
\usepackage{cleveref}

% Geometry allows you to set margins and page dimensions:
\usepackage{geometry}
\geometry{margin=1in}

% For color definitions:
\usepackage{xcolor}

% Fancyhdr for header and footer customization:
\usepackage{fancyhdr}
\pagestyle{fancy}
\fancyhf{}  % Clear all header and footer fields
\fancyhead[LE,RO]{\thepage}         % Page number on left even pages and right odd pages
\fancyhead[LO]{\rightmark}          % Section title on left odd pages
\fancyhead[RE]{\leftmark}           % Chapter title on right even pages

% Microtype improves the overall appearance of the text:
\usepackage{microtype}

%%%%%%%%%%%%%%%%%%%%%%%%%%%%%%%%%%%%%%%%%%%%%%%%%%%%%%%%%%%%%%%%%%%%%
% Theorem Environments and Custom Commands
%%%%%%%%%%%%%%%%%%%%%%%%%%%%%%%%%%%%%%%%%%%%%%%%%%%%%%%%%%%%%%%%%%%%%
% Define theorem-like environments. The numbering is tied to chapters.
\newtheorem{theorem}{Theorem}[chapter]
\newtheorem{lemma}[theorem]{Lemma}       % Shares numbering with theorem
\newtheorem{proposition}[theorem]{Proposition}
\newtheorem{corollary}[theorem]{Corollary}

% For definitions and examples, we switch to a different style:
\theoremstyle{definition}
\newtheorem{definition}[theorem]{Definition}
\newtheorem{example}[theorem]{Example}
% \newtheorem{corollary}[theorem]{Corollary}

% For remarks:
\theoremstyle{remark}
\newtheorem{remark}[theorem]{Remark}

% Some common math shortcuts:
\newcommand{\R}{\mathbb{R}}  % The set of real numbers
\newcommand{\N}{\mathbb{N}}  % The set of natural numbers
\newcommand{\Z}{\mathbb{Z}}  % The set of integers

% definition of new theorem-environments
% \newtheorem{t1}{Definition}
% \newtheorem{t2}[t1]{Theorem}
% \newcommand{t3}[t1]{Corollary}

%%%%%%%%%%%%%%%%%%%%%%%%%%%%%%%%%%%%%%%%%%%%%%%%%%%%%%%%%%%%%%%%%%%%%
% Title and Author Information
%%%%%%%%%%%%%%%%%%%%%%%%%%%%%%%%%%%%%%%%%%%%%%%%%%%%%%%%%%%%%%%%%%%%%
\title{Cauchy, \textit{Course d'analyse} (1821)}
\author{Mattep Bianchetti}
\date{\today}  % Uses the current date; change as needed.

%%%%%%%%%%%%%%%%%%%%%%%%%%%%%%%%%%%%%%%%%%%%%%%%%%%%%%%%%%%%%%%%%%%%%
% Document Body
%%%%%%%%%%%%%%%%%%%%%%%%%%%%%%%%%%%%%%%%%%%%%%%%%%%%%%%%%%%%%%%%%%%%%
\begin{document}

% Front Matter: title page, dedication, preface, table of contents, etc.
\frontmatter
\maketitle

\tableofcontents

% Main Matter: Chapters with the main content
\mainmatter

\chapter*{Preface}
I solve some exercises and prove some statements taken from Cauchy's Course d'analyse (1821).

\section*{Notation}
\begin{enumerate}
    \item $\bigwedge S$ is the least element in the set $S$ (which I assume to be non-empty and strictly ordered).
    \item $\bigvee S$ is the greatest element in the set $S$ (which I assume to be non-empty and strictly ordered).
\end{enumerate}
 
\chapter{Pr\`{e}liminaries}
In the chapter \textit{Pr\`{e}liminaries}, Cauchy discusses the general notion of quantity. He states the following definition and two propositions:

\bigskip

\begin{definition}
    Let $n\in \N$. Given a list of real numbers\footnote{
    ~Cauchy speaks of ``quantities'' (\textit{quantit\'{e}}). I do not claim that Cauchy's quantities coincide exactly with the real numbers.
    }
    $a_1,\ a_2,\ \dots,\ a_n$, a real number $m$ is \textit{medium} (\textit{moyenne}) among the $a_1,\ \dots,\ a_n$ if and only if
    \[
        \bigwedge \{a_1,\ a_2,\ \dots,\ a_n\} \leq m \leq \bigvee \{a_1,\ a_2,\ \dots,\ a_n\}.
    \]
\end{definition}

\bigskip

\begin{theorem}\label{theorem.medium}
Let $n\in \N$. Let $(a_i)_{1\leq i \leq n}$ and $(b_i)_{1\leq i \leq n}$ be sequences of real numbers. Then,
\[
    \frac{\sum_{i=1}^{n} a_1}{\sum_{i=1}^{n} b_i}
\]
is a medium of  
\[
    \frac{a_1}{b_1},\ \dots,\ \frac{a_n}{b_n}.
\]
\end{theorem}

\begin{theorem}
    Let $a_1,\ a_2,\ \dots,\ a_n$ be real numbers. Therefore, 
    \[
        \frac{\sum_{i=1}^{n}a_i}{n}
    \]
    is a medium of $a_1,\ a_2,\ \dots,\ a_n$ and $b_1,\ \dots,\ b_n$ where $b_i = 1$ (and it is called the \textit{arithmetic medium}).
\end{theorem}
One proof (using \cref{theorem.medium}) is the following.
\begin{proof}
    Apply \cref{theorem.medium}.
\end{proof}

The following is a different proof of \cref{theorem.medium} that does not rely on \cref{theorem.medium}.
\begin{proof}
    Let $a = \bigwedge\left\{a_1,\ \dots,\ a_n\right\}$.
Therefore, for some $k_1,\ \dots,\ k_n$, one can rewrite the real numbers in $\left\{a_1,\ \dots,\ a_n\right\}$ as 
\[
    a+k_1,\ \dots,\ a+k_n.
\]
Therefore, 
\[
    a = \frac{\sum_{i=1}^{n} a}{n} \leq \frac{\sum_{i=1}^{n} a}{n} + \frac{\sum_{i=1}^{n} k_i}{n} = \frac{\sum_{i=1}^{n} a+k_1}{n}.
\]

Similarly, for $A = \bigvee \left\{a_1,\ \dots,\ a_n\right\}$, for some $k_1,\ \dots,\ k_n$, one can rewrite $\left\{a_1,\ \dots,\ a_n\right\}$ as 
\[
    A-k_1,\ \dots,\ A-k_n.
\]
Reasoning as above, one shows that 
\[
    \frac{\sum_{i=1}^{n} A-k_1}{n} \leq \frac{\sum_{i=1}^{n} A}{n} = A.
\]
\end{proof}
When $b_i = 1$ for all $1\leq i \leq n$, I will simply say that 
\[
    \frac{\sum_{i=1}^{n} a_i}{n}
\] 
is the \textit{arithmetic mean} of $a_1,\ \dots,\ a_n$ (i.e., I will not mention the $b_i$'s).



% Bibliography (assuming you have a BibTeX file named 'references.bib')
% \bibliographystyle{amsplain}
% \bibliography{references}

\end{document}